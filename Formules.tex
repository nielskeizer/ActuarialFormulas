\documentclass[10pt,landscape]{article}
\usepackage{multicol}
\usepackage{calc}
\usepackage{ifthen}
\usepackage[landscape]{geometry}
\usepackage{amsmath,amsthm,amsfonts,amssymb}
\usepackage{color,graphicx,overpic}
\usepackage{actuarialsymbol, actuarialangle}
\usepackage{bookmark}


\pdfinfo{
  /Title (ARN1 Formules.pdf)
  /Author (Niels Keizer)
  /Subject (Actuarieel Rekenen 1)
  /Keywords (actuariele notatie, samenvatting , formules)}

% This sets page margins to .5 inch if using letter paper, and to 1cm
% if using A4 paper. (This probably isn't strictly necessary.)
% If using another size paper, use default 1cm margins.
\ifthenelse{\lengthtest { \paperwidth = 11in}}
    { \geometry{top=.5in,left=.5in,right=.5in,bottom=.5in} }
    {\ifthenelse{ \lengthtest{ \paperwidth = 297mm}}
        {\geometry{top=1cm,left=1cm,right=1cm,bottom=1cm} }
        {\geometry{top=1cm,left=1cm,right=1cm,bottom=1cm} }
    }

% Turn off header and footer
\pagestyle{empty}

% Redefine section commands to use less space
\makeatletter
\renewcommand{\section}{\@startsection{section}{1}{0mm}%
                                {-1ex plus -.5ex minus -.2ex}%
                                {0.5ex plus .2ex}%x
                                {\normalfont\large\bfseries}}
\renewcommand{\subsection}{\@startsection{subsection}{2}{0mm}%
                                {-1explus -.5ex minus -.2ex}%
                                {0.5ex plus .2ex}%
                                {\normalfont\normalsize\bfseries}}
\renewcommand{\subsubsection}{\@startsection{subsubsection}{3}{0mm}%
                                {-1ex plus -.5ex minus -.2ex}%
                                {1ex plus .2ex}%
                                {\normalfont\small\bfseries}}
\makeatother


% Don't print section numbers
\setcounter{secnumdepth}{0}


\setlength{\parindent}{0pt}
\setlength{\parskip}{0pt plus 0.5ex}

%My Environments
\newtheorem{example}[section]{Example}
% -----------------------------------------------------------------------

\def\annu#1{_{%
\vbox{\hrule height .2pt
\kern 1pt
\hbox{$\scriptstyle {#1}\kern 1pt$}%
}\kern-.05pt
\vrule width .2pt
}}

\begin{document}
\raggedright
\footnotesize
\begin{multicols}{3}


% multicol parameters
% These lengths are set only within the two main columns
%\setlength{\columnseprule}{0.25pt}
\setlength{\premulticols}{1pt}
\setlength{\postmulticols}{1pt}
\setlength{\multicolsep}{1pt}
\setlength{\columnsep}{2pt}

\begin{center}
     \Large{\underline{Formules ARN1}} \\
\end{center}

\section{Kansen}

Kans dat een $x$-jarige binnen 1 jaar overlijdt. 
$$\qx{x}$$

Kans dat een $x$-jarige na $n$ jaar nog leeft.
$$ \px[n]{x} = \prod_{t=0}^{n-1} (1-\qx{x+t}) $$ 

Kans dat een $x$-jarige tussen het $t-1$-de en $t$-de jaar overlijdt.
$$ \qx[t-1|]{x} = \px[t-1]{x} \times \qx{x+t-1} = \px[t-1]{x} - \px[t]{x} $$ 

Kans dat een $x$ en een $y$-jarige na $n$ jaar beiden nog leven
$$ \px[n]{xy} = \px[n]{x} \times \px[n]{y} $$

Kans dat een $x$ en/of een $y$-jarige na $n$ jaar nog in leven is.
$$ \px[n]{\joint{xy}} = \px[n]{x} + \px[n]{y} - \px[n]{xy} $$

Kans dat een het eerste overlijden van een $x$ of $y$-jarige tussen jaar $t-1$ en $t$ is.
$$ \qx[t-1|]{xy} = \px[t-1]{xy} \times ( 1 - \px{x+t-1} \times \px{y+t-1}) = \px[t-1]{xy} - \px[t]{xy} $$

Kans dat van een $x$-jarige en een $y$-jarige, de langstelevende tussen $t-1$ en $t$ overlijdt.
$$ \qx[t-1|]{\joint{xy}} = \qx[t-1|]{x} + \qx[t-1|]{y} - \qx[t-1|]{xy} $$

\section{Verzekeringen}
De actuariele contante waarde van een kasstroomverzekering zonder optionaliteit is te bepalen met:
$$ \sum_{t=0}^{\infty} C(t) \times P(x, t) \times (1 + z_{t})^{-t} $$
Hier is $C(t)$ de kasstroom op tijdstip $t$, $P(x, t)$ is de kans dat de kasstroom wordt uitgekeerd voor verzekerde $x$ op $t$ en $z_{t}$ is een rente termijn structuur.

Eenmalige uitkering bij leven.
$$ \Ex[n]{x} = \px[n]{x} \times (1 + z_{n})^{-n} $$

Levenslange prenumerando lijfrente.
$$ \ax**{x} = \sum_{t=0}^{\infty} \px[t]{x} \times (1 + z_{t})^{-t} $$

Tijdelijke prenumerando lijfrente met duur $n$.
$$ \ax**{\endowxn} = \sum_{t=0}^{n-1} \px[t]{x} \times (1 + z_{t})^{-t} $$

Uitgestelde tijdelijke prenumerando lijfrente met uitstelduur $m$ en betalingsduur $n$.
$$ \ax**[m|]{\endowxn} = \sum_{t=m}^{m+n-1} \px[t]{x} \times (1 + z_{t})^{-t} $$

Levenslange verzekering met uitkering bij overlijden.

$$ \Ax{x} = \sum_{t=0}^{\infty} \qx[t-1|]{x} \times (1 + z_{t})^{-t} $$

Levenslange verzekering met uitkering direct bij overlijden.

$$ \Ax*{x} = \sum_{t=0}^{\infty} \qx[t-1|]{x} \times (1 + z_{t-0.5})^{-t+0.5} $$

\section{Relaties}


% You can even have references
\rule{0.3\linewidth}{0.25pt}
\scriptsize
\end{multicols}
\end{document}
